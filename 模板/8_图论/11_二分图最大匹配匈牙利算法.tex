1、一个二分图中的最大匹配数等于这个图中的最小点覆盖数 \\
König定理是一个二分图中很重要的定理,它的意思是,一个二分图中的最大匹配数等于这个图中的最小点覆盖数。 \\
最小点覆盖:假如选了一个点就相当于覆盖了以它为端点的所有边,你需要选择最少的点来覆盖所有的边。 \\ \\
2、最小路径覆盖=|G|-最大匹配数 \\
在一个$N$*$N$的有向图中,路径覆盖就是在图中找一些路经,使之覆盖了图中的所有顶点,且任何一个顶点有且只有一条路径与之关联。(如果把这些路径中的每条路径从它的起始点走到它的终点,那么恰好可以经过图中的每个顶点一次且仅一次)。如果不考虑图中存在回路,那么每条路径就是一个弱连通子集。 \\
由上面可以得出: \\
1.一个单独的顶点是一条路径 \\
2.如果存在一路径$p_1, p_2, \dots, p_k$,其中$p_1$为起点,$p_k$为终点,那么在覆盖图中,顶点$p_1, p_2, \dots, p_k$不再与其它的顶点之间存在有向边
最小路径覆盖就是找出最小的路径条数,使之成为G的一个路径覆盖。 \\ \\
2、二分图最大独立集=顶点数-最大匹配数 \\
独立集:图中任意两个顶点都不相连的顶点集合。 \\
