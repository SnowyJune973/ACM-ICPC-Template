\section{现场赛宝典}
1、比赛前晚一定要睡眠充足,至少保证8-10个小时睡眠时间。 \\
2、热身赛不要纠结AK什么的,保证A掉一题就可以了。其他时间测试下编译器打表(多长能编译通过)、内存空间、快捷键等等。 \\
3、热身赛每个人都轮流适应一下比赛机器。同时测试一下打印代码的流程,和提问、提交流程。要求每个队员都熟悉流程,保证提交不会出现错误。 \\
4、热身赛后及时调节心情,如果一定要想出这道题怎么做,要果断问身边的大牛,没有疑惑和纠结度过周六。 \\
5、赛前的心情非常非常重要!队友之间一定要互相开玩笑,保持愉快的心情开场。实在所有人都没有心情娱乐,那就脸部保持微笑状态三分钟,保证能改变心情。(强颜欢笑也是乐) \\
6、开场后分配好题目(一般前中后三部分),英语好要迅速读完题目,并且指定一个人每两三分钟看一次rank,保证FB后迅速换题) \\
7、如果读题过程中能确定某题可出,并且为队友的擅长题型,迅速通知队友,并说清题意、题目条件,讨论是否可敲。如果敲题,这时候读完自己题目后,还继续把队友没读的题读完。 \\
8、看到有人出题后一定要第一时间阅读该题,有条件最好两人同时读,并且保证题目描述、条件和输入输出都没有异议。讨论该题的可行性,如果完全没有思路,并且是神校出题,再观望一段时间,别盲目敲题。 \\
9、一定一定要记住!比赛有五个小时,开场一定要淡定,保证好首次提交,很影响士气。两人仔细检查输入输出范围,多跑几遍样例,保证多case考虑了,保证输出格式、没多余空格、空行。可能的trick还有提交流程。 \\
10、提交后有一段等待时间,这时候不要全部盯着屏幕等,不管对错,直接打印代码,然后再检查一遍输入输出和数据,检查代码有没有问题。空闲的人继续读题或看榜。 \\
11、如果提交没过,有其他题在开着,换人敲。之前的人拿着打印的代码检查bug。如不能保证bug很快找到,至少两个人同时debug,不需要双开。 \\
12、如果某题2次WA,并且找不到原因,这时候一边debug一边让别人重新读一遍题目,逐字分析条件和输入输出,不受其他人影响,然后交流题意等等是否出现问题。题意和算法没问题的情况下,想trick出数据。 \\
13、YES了一道题后,主敲队员最好的放松方法就是去趟洗手间!不管想不想方便,去洗个脸,呼吸一下新鲜空气也是很有必要的。清醒一下,把YES的那道题抛到脑后,关注其他题。还有就是在脑子非常乱的时候,非常憋屈的时候,也去趟洗手间,说不定还能偷瞄到大神的思路…… \\
14、比赛士气和心情很重要,一定要有一个队员心理素质够硬,负责调动全队状态。小口喝水、身体坐正,双手交叉,互相打气能快速帮你调整状态。 \\
15、比赛期间以观察rank为出题标准,但是如果认定该题不适合你们,要果断先选择另外一道过的人较多的题! \\
16、在没有两个人都足够主敲的情况下,切忌双开敲题,尤其到了最后一两个小时,一般三人全力攻题!出题的同时,空闲队员想好trick数据等等。 \\
17、封榜后一定不能乱,不要盲目交题,不要重复交同样代码!!参考11、12 \\
18、记得比赛比的是心态和状态,如果浮躁,自己就先输了,一定要淡定!如果开始自己或队友发现出现浮躁慌乱,一定要提醒大家,参考13、14 \\
19、相信自己,相信队友,和谐相处,共同拿牌!!! \\
20、欢迎补充~~ \\
