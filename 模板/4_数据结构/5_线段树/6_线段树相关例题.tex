\subsubsection{线段树相关例题}

单点增加,区间求和
\lstinputlisting{"./模板/4_数据结构/5_线段树/6_线段树相关例题/单点更新/1_HDU1166.cpp"}
单点修改,区间求最值
\lstinputlisting{"./模板/4_数据结构/5_线段树/6_线段树相关例题/单点更新/2_HDU1754.cpp"}
%单点增加,区间求和(求逆序数)
%\lstinputlisting{"./模板/4_数据结构/5_线段树/6_线段树相关例题/单点更新/3_HDU1394.cpp"}
单点找最大值$\ge$c的第一个位置,同时单点更新
\lstinputlisting{"./模板/4_数据结构/5_线段树/6_线段树相关例题/单点更新/4_HDU2795.cpp"}
找第K大数,同时单点更新。(使用sum数组)
\lstinputlisting{"./模板/4_数据结构/5_线段树/6_线段树相关例题/单点更新/5_POJ2828.cpp"}
%找第K大数,同时单点更新。(使用sum数组)
%\lstinputlisting{"./模板/4_数据结构/5_线段树/6_线段树相关例题/单点更新/6_POJ2886.cpp"}
1、add x表示往集合里添加数x。2、del x表示将集合中数x删除。3、sum求出从小到大排列的集合中下标模$5$为$3$的数的和。集合中的数都是唯一的。\\
在线段树中维护当前这个集合中数的个数sum,和所有的数模$5$为$0\cdots4$内的数的和设为ans[$0\cdots4$]。在进行区间合并的时候,父区间里的ans[$0\cdots4$]首先等于左子区间里的ans[$0\cdots4$],设要加入的右子区间的数为ans[i],则它应该加到父区间的ans[sum[rt<<1]+i]。
\lstinputlisting{"./模板/4_数据结构/5_线段树/6_线段树相关例题/单点更新/7_HDU4288.cpp"}
有三种操作“add x y”往平面上添加(x,y)这个点,“remove x y”,将平面上已经存在的点(x,y)删除,“find x y”找出平面上坐标严格大于(x,y)的点,如果有多个点找x最小的,再找y最小的。\\
用线段树维护y的最大值。
\lstinputlisting{"./模板/4_数据结构/5_线段树/6_线段树相关例题/单点更新/8_CF19D.cpp"}
%给你一些区间,问对于给出的每个区间,有多少个区间是完全包含它的。其实就是求逆序数。
%\lstinputlisting{"./模板/4_数据结构/5_线段树/6_线段树相关例题/单点更新/9_POJ2481.cpp"}
%单点增加,求第K大数
%\lstinputlisting{"./模板/4_数据结构/5_线段树/6_线段树相关例题/单点更新/ZOJProblemSet-3612.cpp"}


区间置值,区间求和
\lstinputlisting{"./模板/4_数据结构/5_线段树/6_线段树相关例题/区间更新/1_HDU1698.cpp"}
%区间增加,区间求和
%\lstinputlisting{"./模板/4_数据结构/5_线段树/6_线段树相关例题/区间更新/2_POJ3468.cpp"}
区间置值,单点询问
\lstinputlisting{"./模板/4_数据结构/5_线段树/6_线段树相关例题/区间更新/3_POJ2528.cpp"}
区间置值,区间查询有几种值
\lstinputlisting{"./模板/4_数据结构/5_线段树/6_线段树相关例题/区间更新/4_POJ1436.cpp"}
有N根杆子,前后两根杆子通过一个节点连接,每个节点可以旋转一定的角度,每次给你一个操作(s,a)表示将第S与S+1之间的角度修改为a度,并且每次操作之后都需要求出第N个节点的位置。\\
首先,最后一个节点的坐标(即位置)可以通过所有杆子末尾的坐标相加得到(这跟向量相加很类似)。因为每次更新需要修改S和S+1的位置,所以我们需要知道每次更新之后S和S+1的角度——查询操作,知道之后,再根据这次需要修改的角度a,得到S+1到最后一段杆子需要改变的角度——更新操作。最后输出答案即所有节点的x坐标和,y坐标和。\\
add为ang增加的lazy标记,ang记录杆子末尾的角度,每次更新前query角度,就可以知道此次更新的角度的相对值,从而修改坐标x,y。
\lstinputlisting{"./模板/4_数据结构/5_线段树/6_线段树相关例题/区间更新/5_POJ2991.cpp"}
给定n长的序列 m个操作\\
序列默认为 1, 2, 3...n\\
操作1:D [l,r] 把[l,r]区间增长 :( 1,2,3,4 进行 D [1,3]变成 1,1,2,2,3,3,4 )\\
操作2:Q [l,r] 问区间[l,r] 上出现最多次数的数的次数\\
区间乘积,单点增加,区间求最值,找第K大数
\lstinputlisting{"./模板/4_数据结构/5_线段树/6_线段树相关例题/区间更新/HDU4973.cpp"}
区间更新最值,单点查询最值
\lstinputlisting{"./模板/4_数据结构/5_线段树/6_线段树相关例题/区间更新/ZOJProblemSet-3632.cpp"}
