\section{心得总结}
  经过一个暑假全身心投入的集训,我的收获挺大的,算法水平得到了全面的提升,下面我来总结一下。\\
  我主要专攻数据结构部分,首先强化了之前学习的树状数组、线段树等,让我对它们有了更深的理解,接着学习了很多强大的数据结构,如可并堆左偏树,效率很高的平衡二叉搜索树Size Balanced Tree,处理区间反转的splay树,查询区间第K大树的划分树,查询区间第K大树同时支持更新的函数式线段树,求点到点集最近的点的KD-tree等等,让我对数据结构有了更深刻的认识。尤其是splay树,它独特的伸展操作真的是相当巧妙,每次查询和更新之后都执行一次伸展操作,这样就可以使均摊复杂度降低到$O(n\log n)$,还有快速的拆分和合并操作,这是其它结构较难实现的。\\
  字符串方面,我深入学习了哈希,感受到哈希在字符串处理方面的强大,而且用某种特定的哈希方法就可以基本保证不会冲突,尤其是BKDRHash哈希法,不仅哈希方法简单,而且冲突率低,不得不佩服那些计算机科学家的智慧。\\
  当然,其他方面我也学习了很多,例如在图论方面,强化了基本的算法例如最短路、最小生成树,学习了网络流以及强化了建图方法。数论方面,对各种定理有了更深刻的理解。动态规划方面,学习了各种子序列问题。动态规划主要在于其思想,最重要的就是推出状态转移方程,然后是边界条件的处理。这需要长时间的积累,短时间提升的难度还是挺大的。还有就是C++的STL,已经可以熟练使用各种STL函数,这对快速解题是有很大帮助的。最后是一些要整理成模板的东西,比如高精度整数类、分数类、矩阵类、自适应Simpson积分法以及常数优化的相关代码等,有一个好的模板,在比赛中可以节约很多时间。\\
  这个暑假,我觉得我和队友的状态都还不错,我们的水平也都得到了很大的提升,今年已经打完的三场网络赛成绩也都不错,希望这样的势头可以保持下去,在现场赛争金夺银,不要留下遗憾!\\